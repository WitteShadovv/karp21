\section{Max-Cut}

Eine Firma konnte in einer feindlichen Übernahme einen wichtigen Konkurrenten aufkaufen. Der Konkurrent ist eigentlich eine Vereinigung von sehr vielen, sehr intensiv und effizient zusammenarbeitenden, aber im Übrigen weitgehend selbständigen Abteilungen, alle unter dem selben Dach. Das Ziel der Übernahme war, daraus die “Rosinen” herauszupicken, und den Rest zu schliessen. Die Wettbewerbsbehörden hatten dies vorausgesehen, und als Auflage für die Übernahme gemacht, dass keine einzige der Abteilungen geschlossen werden dürfe. Daher dachten sich die neuen Eigentümer den folgenden bösartigen Plan aus, um den gleichen Zweck zu erreichen. Sie teilten die Abteilungen auf zwei verschiedene Standorte auf, und sorgten dafür, dass jede Kommunikation zwischen den beiden Standorten so ineffizient wurde, dass die Abtei- lungen kaum mehr sinnvoll zusammenarbeiten konnten. Dadurch würden die einzelnen Abteilungen wirtschaftlich ruiniert, und man müsste sie trotz allem schliessen. Die neuen Eigentümer beauftragten daher eine Beratungsfirma, eine Aufteilung zu finden, mit der die Kommunikation zwischen den Abteilungen möglichst stark behindert würde. Die Beratungsfirma brauchte dafür sehr lange. Warum ist das nicht überraschend?

Es handelt sich hier um das Problem MAX-CUT. Wir beschreiben eine Reduktion des Problems auf MAX-CUT:

\begin{itemize}
\item Abteilung $\leftrightarrow$ Vertex
\item Kommunikationsbeziehung $\leftrightarrow$ Kante
\item Kommunikationsvolumen $\leftrightarrow$ Gewicht einer Kante
\end{itemize}

Die neuen Firmeneigentümer wollen die Menge der Vertices so in zwei Mengen aufteilen, dass die Summe der Gewichte der Kanten, die durch die Aufteilung zerschnitten werden, möglichst gross wird. Dies ist genau die Beschreibung des Problems MAX-CUT. Das Problem MAX-CUT ist NP-vollständig, nach aktuellem Wissen gibt es also keinen effizienten (polynomiellen) Algorithmus, der ein MAX-CUT Problem lösen könnte.