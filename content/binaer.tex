\section{Binär}
\subsection{Begriffe}
\textcolor{myblue}{Bit}: Stelle einer Binärzahl\\
\textcolor{myblue}{Bit = 1}: Set Bit\\
\textcolor{myblue}{Bit = 0}: Gelöschtes Bit (cleared bit) \\
\textcolor{myblue}{LSB}: Least Significant Bit, niederwertigstes Bit, Bit 0\\
\textcolor{myblue}{MSB}: Most Significant Bit, höchstwertiges Bit, Bit n-1\\
\textcolor{myblue}{Nibble}: Binärzahl mit vier Bit.\\
\textcolor{myblue}{Byte/Oktett}: Binärzahl mit acht Bit.\\
\textcolor{myblue}{Carry Bit}: Übertragsbit, wird bei einem Übertrag gesetzt.\\

\subsection{Formeln}
\textcolor{myblue}{Grösste Darstellbare Zahl(unsigned)}: 2n-1 \\
\textcolor{myblue}{Anzahl Zahlen}: 2n \\
\textcolor{myblue}{Bereich(unsigned)}: 0 bis 2n-1 \\
\textcolor{myblue}{Grösste Darstellbare Zahl(signed)}: 2n-1-1 \\
\textcolor{myblue}{Kleinste Negative Zahl (signed)}: -2n-1 \\
\textcolor{myblue}{Bereich(signed)}: -2n-1 bis 2n-1-1 \\
\textcolor{myblue}{MSB = 0}: Dient als positives Vorzeichen \\
\textcolor{myblue}{MSB = 1}: Dient als negatives Vorzeichen 
\subsection{Operationen}
\textcolor{myblue}{Muliplikation}\\
Wie schriftliche Multiplikation im Dezimalsystem.\\
\textcolor{myblue}{2er Potenz über Binar zu Hex}\\
$2^{12}_d\rightarrow 0001 \,  0000 \, 0000 \, 0000 \rightarrow 1000_h$\\
\textcolor{myblue}{Invertieren}\\
$0001 \,  0000 \, 0000 \, 0000 \rightarrow 1110 \, 1111 \, 1111 \, 1111 \,$\\
\textcolor{myblue}{Zweierkomplement}\\
Durch MSB als Vorzeichen können Binärzahlen negativ dargestellt werden.\\
\textcolor{myblue}{Zweierkomplement Binär}:Invertieren und +1\\
$0001 \,  0000 \, 0000 \, 0000 \rightarrow 1110 \, 1111 \, 1111 \, 1111$\\
$\rightarrow 1111 \, 0000 \, 0000 \, 0000$\\
\textcolor{myblue}{Zweierkomplement in Dezimal}: MSB abziehen den Rest addieren\\
$1101_b \rightarrow -8*1+4*1+2*0+1*1 = -3$\\
\textcolor{myblue}{Links- \& Rechtsshift}\\
\textcolor{myblue}{Rechtsshift}: Wenn negativ dann 1 nachschieben, sonst 0\\
Bsp negativ: $1010 \rightarrow 1101$\\
Bsp positiv: $0110 \rightarrow 0011$\\
\textcolor{myblue}{Linksshift}: Es wird immer eine 0 von rechts nachgeschoben.\\
Bsp:  $0110 \rightarrow 1100$\\
