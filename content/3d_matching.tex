\section{3D-Matching}
\subsection{Konzertprogramm}
Ein Orchester soll Konzertprogramme für eine Saison zusammenstellen. Zur Verfügung steht eine Menge von Werken, die das Orchester im Repertoire hat. Aus diesen Werken sollen Programme mit jeweils drei Werken zusammengesetzt werden. Je ein Werk zur Eröffnung, für den Mittelteil und zum Abschluss des Konzertes. Die Werke lassen sich nicht beliebig kombi- nieren, weil zum Beispiel die Tonarten nicht zu verschieden sein dürfen. Die Anzahl der Werke und der Konzertabende der Saison ist gleich. Es stellt sich heraus, dass es ziemlich schwierig ist, Programme so zu konstruieren, dass kein Werk in der gleichen Phase eines Konzerts mehr als einmal gespielt wird. Warum?\\
\\
\textbf{Lösung}
Dies ist eine Instanz des 3D-Matching-Problems. Die Menge der Werke ist T , die Menge der möglichen Programme (Einschränkungen zum Beispiel durch Tonart) ist U ⊂ T × T × T . Gesucht werden solle eine Menge von Programmen W ⊂ U, mit |W| = |T |, so dass keine zwei Elemente in irgend einer Koordinate übereinstimmen. Da 3D-Matching NP-vollständig ist, ist auch das Konzertproblem NP-vollständig, nach aktuellem Wissen gibt es dafür daher keinen polynomiellen Algorithmus.
\subsection{Rezepte}
Ein Koch hat je n Rezepte für Vorspeisen, Hauptspeisen und Desserts. Nicht alle Vorspeisen lassen sich mit jeder Hauptspeise kombinieren, dasselbe gilt auch für Desserts. Damit jedes seiner Rezepte regelmässig zum Einsatz kommt, möchte der Koch eine Folge von n Menus zusammenstellen, so dass jedes Rezept in genau einem der Menus vorkommt. Nach längerem tüfteln gibt er jedoch frustriert auf. Können Sie erklären, warum ihm die Menugestaltung so schwer gefallen ist.\\
\\
\textbf{Lösung}
Es handelt sich hier um das Problem 3D-MATCHING. Die Menge T sind die Nummern der Rezepte, die Tripel aus 
T × T × T sind die Menuzusammenstellungen, bestehend aus je einer Nummer für ein Vorspeisen-, ein Hauptspeisen- und ein Dessert-Rezept. Die Menge U ist die Menge der möglichen Menukombinationen. Die gesuchte Teilmenge W ist eine Auswahl von n = |T | = |W| Menus derart, dass keine Vorspeise (erste Komponente), keine Hauptspeise (zweite Komponente) und kein Dessert (dritte Komponente) mehr als einmal vorkommt. Das Problem 3D-MATCHING ist NP-vollständig, es ist daher kein Algorithmus bekannt, der das Problem in polynomieller Zeit lösen könnte.