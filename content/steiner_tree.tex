\section{Steiner-Tree}
In einem Entwicklungsland sollen die aus dem Ausland erhaltenen Unterstützungsmittel dazu verwendet werden, endlich alle Ortschaften mit mindestens 100 Einwohnern ans Stromnetz anzuschliessen. Der Bau von Leitungen zwischen einzelnen Ortschaften ist je nach Gelände unterschiedlich teuer, zum Teil auch schlicht unmöglich. Es wird entschieden, dass man in einer ersten Phase auf Redundanz des neu zu erstellenden Netzes verzichten will. Der Minister möchte endlich wissen, ob das vorhandene Geld für das Projekt ausreicht, und ist sehr ungehalten darüber, dass die Verwaltung so lange braucht, diese Frage zu beantworten. Kann man dies erklären?\\
\\
\textbf{Lösung} Dieses STROMNETZ genannte Problem ist äquivalent zu STEINER-TREE wie folgt. Die Ortschaften des Landes entsprechen der Menge \textbf{$V$} aller Vertizes des Graphen. Die untereinander zu verbindenden Ortschaften bilden eine Teilmenge \textbf{$R \subset V$}. Eine Kante im Graphen \textbf{$G$} entspricht zwei verbindbaren Ortschaften, jeder solchen Kante sind die Kosten für den Bau einer Verbindungsleitung zugeordnet. Es ist nun eine Menge von Verbindungsleitungen zu wählen, die einen Baum (keine Redundanz) bilden und so, dass die Summe der Gewichte das Budget nicht übersteigt.
\begin{itemize}
\item STEINER-TREE $\leftrightarrow$ STROMNETZ
\item Knoten $\leftrightarrow$ Ortschaften
\item Knoten in R $\leftrightarrow$ zu erschliessende Ortschaften
\item Gewicht w einer Kante $\leftrightarrow$ Baukosten einer Verbindungsleitung
\item maximales Gewicht k $\leftrightarrow$ Budget
\end{itemize}

