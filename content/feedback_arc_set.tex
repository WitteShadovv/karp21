\section{Feedback-Arc-Set}
\subsection{COVID-19 Sicherheitsmassnahmen}
Die COVID-19 Pandemie verlangt von Läden Sicherheitsmassnahmen, die sicherstellen sollen, dass die Kunden in sicheren Abständen verbleiben. Ein Einkaufszentrum entscheidet daher, dass ein Einbahnbetrieb eingeführt werden soll. Die Marketing-Abteilung legt fest, in welcher Richtung die einzelnen Gänge zwischen den Gestellen durchlaufen werden können sollen. Es soll nämlich sichergestellt werden, dass die Kunden immer noch in der “richtigen” Reihenfolge mit überflüssiger Werbung zu genauso überflüssigen Spontankäufen verleitet werden sollen. Schliesslich will der Sicherheitsverantwortliche wissen, ob es möglich ist, in einigen Gängen Desinfektionsstationen aufzustellen so, dass jeder Kunde, der sich auf einem geschlossenen Weg durch das Einkaufszentrum bewegt, mindestens einmal an einer Desinfektionsstation vorbeikommt. Dabei dürfen nicht mehr Desinfektionsstationen verwendet werden, als das vorgegebene Budget erlaubt. Dies stürtzt die Planer in eine Krise, auch nach stundenlangem Probieren können Sie keine definitive Antwort geben. Warum?\\
\\
\textbf{Lösung}
Das beschriebene Problem ist das Problem FEEDBACK-ARC-SET, wie die folgenden
Eins-zu-eins-Reduktion zeigt:\\
Graph $\leftrightarrow$ Plan des Einkaufszentrums\\
Knoten $\leftrightarrow$ Kreuzungsstellen\\
Kanten $\leftrightarrow$ Gänge\\
Richtung $\leftrightarrow$ Einbahnrichtung in jedem Gang\\
Anzahl k $\leftrightarrow$ Budget für Desinfektionsstationen\\
Arc set $\leftrightarrow$ Platzierung der Desinfektionsstationen