\section{Assembler}
Programm, welches textuelle Befehle in Maschinencode übersetzt. \\
Die Sprache dazu heisst Assembly. Die Konventionen zu Assembly sind jeweils abhängig vom Hersteller.\\

\textcolor{myblue}{Befehlssatz:} Menge aller Maschinencodes, die ein Prozessor kennt.\\

Diese Maschinencodes können unterschiedlich lang sein:
\begin{itemize}
  \item 1 Byte (8 Bit): byte, DB, RESB --> hat 2 Hexadez. Stellen
  \item 2 Byte: word, DW, RESW
  \item 4 Byte: dword, DD, RESD
  \item 8 Byte: qword, DQ, RESQ
\end{itemize}

\subsection{Register}
\textcolor{myblue}{AL} 8 Bit Register\\
\textcolor{myblue}{AX}: 16 Bit Register mit je zwei 8 Bit Registern (AH (oben), AL (unten))\\
\textcolor{myblue}{EAX}: 32 Bit Register, erweitert AX links mit 16 Bit\\
\textcolor{myblue}{RAX}: 64 Bit Register, erweitert EAX links mit 32 Bit\\

\subsection{Allzweckregister}
\textcolor{myblue}{RAX}: Accumulator, für einige Rechenoperationen das einzige Register\\
\textcolor{myblue}{RCX}: Counter für Schleifen und Stringoperationen\\
\textcolor{myblue}{RDX}: Pointer für I/O-Operationen\\
\textcolor{myblue}{RBX}: Datenpointer\\
\textcolor{myblue}{RSI / RDI}: Quell- und Zielindizes für Stringoperationen\\
\textcolor{myblue}{RSP}: Stackpointer, Adresse des allozierten Stacks\\
\textcolor{myblue}{RBP}: Basepointer, Adresse innerhalb des Stacks, Basis des Rahmens der Funktion\\
\textcolor{myblue}{R8 – R15}: Zusätzliche Register\\
\subsection{Endians}
\textcolor{myblue}{Little-Endian}: BE | BA | FE | CA	--> innerhalb des Bytes bleibt es gleich, aber das LSB (Least Significant Bit) ist am «Ende»\\
\textcolor{myblue}{Big-Endian}: CA | FE | BA | BE --> unsere "normale"\ Art. MSB (Most Significant Bit) steht am «Ende».
\subsection{Labels}
Am Anfang jeder Zeile kann ein sogenanntes Label stehen, welches nicht in Bytecode übersetzt und ausgegeben wird. Intern assoziiert der Assembler Speicher für das Label.
\subsection{Syntax}
\textcolor{myblue}{\_myfunction}: Definiert \_myfunction als Pointer auf die Stelle im generierten Byte-Stream\\
\textcolor{myblue}{mov x, y}: Kopiert den Wert vom Register rbx in rax\\
Bsp: mov cl, [rax] –> es werden 8 Bit bewegt.\\
\textcolor{myblue}{mov x, [y]}: Kopiert das Byte nach x, das an der Hauptspeicheradresse liegt,
die in y steht.\\
\textcolor{myblue}{inc rax}: Erhöht Wert in rax um 1\\
\textcolor{myblue}{dec rax}: Verringert Wert in rax um 1\\
\textcolor{myblue}{cmp x, y}: Vergleicht x mit y und setzt Z(ero)-Flag, wenn beide gleich sind.\\
\textcolor{myblue}{sub x, y}: y wird von x abgezogen und Differenz in x geschrieben.\\
\textcolor{myblue}{jnz \_myfunction}: Springt zu \_myfunction wenn Z-Flag nicht gesetzt.\\
\textcolor{myblue}{ret}: Return, Rücksprung zum Aufrufer und vorher Rücksprungadresse vom Stack holen und in Befehlszähler schreiben.
\subsection{Calling Convention}
Calling Convention sind Vereinbarungen zwischen dem Caller und
der aufgerufener Funktion(Callee): Wo Argumente, Wo Rückgabewerte, welche
Register bearbeitet, etc.\\
\subsection{Lokal \& globale Variablen}
\textcolor{myblue}{Lokale Variable}: Liegt auf dem Stack\\
\textcolor{myblue}{Globale Variable}: Fixe Adresse im Speicher(Label)